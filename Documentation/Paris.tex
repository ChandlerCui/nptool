\documentclass[a4paper,12pt]{article}
\usepackage[T1]{fontenc}
\usepackage [isolatin]{inputenc}
\usepackage{epsfig}
\usepackage{graphicx}
\usepackage{listings}

\begin{document}

\title{\emph{PARIS Documentation}}
\author{Marc Labiche\\ \\
 STFC Daresbury Laboratory,\\
 marc.labiche@stfc.ac.uk\\ \\} 

\maketitle 
\pagebreak
\tableofcontents
\pagebreak


\section{Introduction}
The PARIS project is a new $\gamma$-ray array which aims at addressing topical nuclear 
physics research programs with SPIRAL2 beams.
We describe this project here as it is developed within the simulation GEANT4-based 
NPTool framework. The design and choice of the scintillator is still matter of 
discussion within the collaboration, and several and independent simulation frameworks 
are being used to help converging towards a definitive solution. In NPTool, we consider 
an array of phoswich detectors consisting of LaBr$_{3}$ and NaI (originally CsI) 
scintillators, surrounded by NaI (originally CsI) shields, in an EXOGAM-like spherical 
configuration. The modular 
aspect of the NPTool framework allows for the PARIS array to be modelled either as a 
standalone detector or coupled to other NPTool modelled detectors like GASPARD and 
MUST2.The general structure of the PARIS program is based on the structure of the 
GASPARD program, thus many details described here are similar to the GASPARD document.

\section{NPSimulation}
\subsection{Specificity of Paris}

\begin{figure*}[ht]
\begin{center}
\psfig{figure=Parismodule1.eps,width=10cm,height=5cm,angle=0}
\end{center}
\caption{Two drawings of a single phoswich module. The LaBr$_{3}$ scintillator is in blue 
and the NaI scintillator is in red } 
\label{fig:phoswich1}
\end{figure*}

In this released of NPTool, the PARIS $\gamma$-ray array consists of phoswich detectors 
with squared cross section of 2''$\times$2''. The phoswich first layer is a 2'' long 
LaBr$_{3}$ scintillator. The second layer is a 6'' long scintillator of NaI scintillator 
(see fig.~\ref{fig:phoswich1}). Originally, CsI scintillators were considered and, for 
that unique reason, many comments and variable names in the program still point at CsI 
scintillators. The user should be aware that the CsI material has been replaced by NaI 
following prototype testing results. All the comments and variable names will be 
corrected in future a release. 
  
 The phoswich modules are grouped in 3$\times$3 clusters (see fig.~\ref{fig:phoswich3by3})  
or used as single modules (see fig.~\ref{fig:phoswich1}) . All are positioned to form a 
spherical configuration of an adjustable inner radius between 20.8 and 23.5 cm, as show in 
fig.~\ref{fig:Paris1}. A shield made of extra NaI scintillators surrounds the cluster and 
single phoswich to fill the gaps in between, and give a spherical geometry to the full array 
(see fig.~\ref{fig:Paris2}). The clusters and single phoswich modules of the Paris detector 
are described in the Paris class defined in the Paris.\{hh,cc\} files. The geometry of the 
cluster is defined in the ParisCluster.\{hh,cc\} files, and the geometry of a single phoswich
module is defined in the ParisPhoswich.\{hh,cc\} files.
The shields of CsI scintillators are described in a different class called Shield, and 
defined in the Shield.\{hh,cc\} files. Different shield geometries are used for the clusters 
and single phoswich modules. The geometry of the cluster shield is defined in the files 
ShieldClParis.\{hh,cc\}. The geometry of the single phoswich module shield is defined in the 
files ShieldPhParis.\{hh,cc\}.

\begin{figure*}[ht]
\begin{center}
\psfig{figure=Paris3by3.eps,width=12cm,height=5cm,angle=0}
\end{center}
\caption{Two drawings of a 3$\times$3 cluster. The LaBr$_{3}$ scintillator is in blue and 
the NaI scintillator is in red } 
\label{fig:phoswich3by3}
\end{figure*}
  
Since the Paris detector and the NaI shield are registered in the DetectorConstructor.cc file
,they are available for NPSimulation.

As for other NPTool modelled detectors, in order to manage the different detector shapes 
(cluster, single module, cluster shield, single module shield) of the Paris array, the Paris 
and Shield classes hold a vector of ParisModule and ShieldModule objects, respectively, from 
which are deriving all the different shapes (ParisCluster, ParisPhoswich,ShieldClParis, and 
ShieldPhParis classes).

\begin{figure*}[ht]
\begin{center}
\psfig{figure=FullParisNoShield.eps,width=8cm,height=8cm,angle=0}
\end{center}
\caption{Spherical configuration of Paris, without shielding.} 
\label{fig:Paris1}
\end{figure*}

\begin{figure*}[ht]
\begin{center}
\psfig{figure=FullParis.eps,width=8cm,height=8cm,angle=0}
\end{center}
\caption{Spherical configuration of Paris, with shielding.} 
\label{fig:Paris2}
\end{figure*}

\subsection{Running the simulation}
As for other NPTool simulations, to run Paris simulations the following command line should 
be executed: 

\begin{verbatim}
   Simulation -D yyy.detector -E  xxx.reaction
\end{verbatim}

where xxx.reaction is an input file describing the event generator and
yyy.detector is an input file describing the detector geometry. All these
input files are based on keywords and can be found in the 
\$NPTool/Inputs subdirectories.

\subsubsection{Event Generators}
In the distributed version, a source of $\gamma$-ray either at rest or in motion along the 
beam axis can be used. For an isotropic or a collimated source at rest, the event generator 
file isotropic.source in the Inputs/EventGenerator is available.An example is given in the 
file gamma.source. For a source in the move, one can use the example given in the 
source.reaction file.  

\subsubsection{Detector Configurations}
The keywords associated to the detector geometry file are different for 
each detector. In case of the Paris detector, an example with 
all the kind of detector shapes available at the moment is given below:

\begin{verbatim}
   %%%%%%%%%%%%%%%%%%%%%%%%%%%%%%%%%%%%
   Paris
   %%%%%%%%%%%%%%%%%%%%%%%%%%%%%%%%%%%%
   ParisCluster
     X1_Y1=       -84.5	-208   84.5
     X1_Y128=      84.5	-208   84.5
     X128_Y1=     -84.5	-208  -84.5
     X128_Y128=    84.5	-208  -84.5
     VIS=all
   %%%%%%%%%%%%%%%%%%%%%%%%%%%%%%%%%%%%
   ParisPhoswich
     X1_Y1=      -128.6063166   143.3590149   -88.301235
     X1_Y128=    -151.8764696    96.81870993 -111.571389
     X128_Y1=     -88.30122956  143.3590149  -128.606323
     X128_Y128=  -111.5713826    96.81870993  -151.876476
     VIS= all
   %%%%%%%%%%%%%%%%%%%%%%%%%%%%%%%%%%%%
   Shield
   %%%%%%%%%%%%%%%%%%%%%%%%%%%%%%%%%%%%
   ShieldClParis
     X1_Y1=       151.375  153.75   364
     X1_Y128=     151.375   91.5    364
     X128_Y1=     -84.5    153.75   364
     X128_Y128=   -84.5     91.5    364
     VIS= all
   %%%%%%%%%%%%%%%%%%%%%%%%%%%%%%%%%%%%
   ShieldPhParis
     THETA= 54.73
     PHI=   45.02
     R=	   248
     BETA=   0 0 -30
     VIS= all
\end{verbatim}

The Paris and shield detectors are independent. In other words, one can use Paris detectors 
with or without the NaI shields. 

In order to declare a Paris detector in NPSimulation, the key
word {\it Paris} should be specified in the geometry file. It
should then be followed by other keywords concerning the different
detectors present. Such keywords available at the moment are:

\begin{itemize}
   \item {ParisCluster}
   \item {ParisPhoswich}
\end{itemize}

Each keyword corresponds to a detector shape which has its own set of keywords which is used 
to position the detector in the world volume.
For the Shield detectors, the keyword {\it Shield} should be specified in the geometry file, 
followed by other keywords associated to the different detector shield for cluster (Cl) and 
single phoswich (Ph):

\begin{itemize}
   \item {ShieldClParis}
   \item {ShieldPhParis}
\end{itemize}

In principle, to position the detectors two possibilities exist. Either the Cartesian 
coordinates (x,y,z) of each detector's corner are specified with the keywords {\it X1\_Y1}, 
{\it X1\_Y128}, {\it X128\_Y1} and {\it X128\_Y128}(case of ParisCluster, ParisPhoswich and 
ShieldClParis), either the spherical coordinates of the detector's centre are specified with 
the keywords {\it R}, {\it THETA} and {\it PHI} (case of ShieldPhParis). While the first 
solution is very helpful when working with the mechanical engineers, the second solution is 
useful when investigating new geometries. However, up to now, only one solution for each 
type of Paris and Shield detectors has been tested, and this is the solution given above as 
an example.

\subsubsection{Physics list}

The electromagnetic physics process used by default in NPTool is the standard GEANT4 physics 
list. The user can  however choose a different physics list amongst the Standard, Low Energy 
and Penelope physics list by modifying the file: \$NPTool/NPSimulation/src/PhysicList.cc.
As an example,in order to choose the low energy physics list (recommended), open the 
PhysicList.cc file and comment and uncomment the relevant lines as follow:

\scriptsize
\begin{verbatim}
  if (particleName == "gamma") {
    // gamma
    //standard Geant4
    //pmanager->AddDiscreteProcess(new G4PhotoElectricEffect);
    //pmanager->AddDiscreteProcess(new G4ComptonScattering);
    //pmanager->AddDiscreteProcess(new G4GammaConversion);

    //Low energy
    //pmanager->AddDiscreteProcess(new G4LowEnergyPhotoElectric);
    pmanager->AddDiscreteProcess(new G4LowEnergyCompton);
    G4LowEnergyPhotoElectric* LePeprocess = new G4LowEnergyPhotoElectric();
    LePeprocess->ActivateAuger(true);
    LePeprocess->SetCutForLowEnSecPhotons(0.250*keV);
    LePeprocess->SetCutForLowEnSecElectrons(0.250*keV);
    pmanager->AddDiscreteProcess(LePeprocess);
    pmanager->AddDiscreteProcess(new G4LowEnergyGammaConversion);
    pmanager->AddDiscreteProcess(new G4LowEnergyRayleigh);
    pmanager->AddProcess(new G4StepLimiter(),-1,-1,3);

    // Penelope
    //pmanager->AddDiscreteProcess(new G4PenelopePhotoElectric);
    //pmanager->AddDiscreteProcess(new G4PenelopeCompton);
    //pmanager->AddDiscreteProcess(new G4PenelopeGammaConversion);
    //pmanager->AddDiscreteProcess(new G4PenelopeRayleigh);

  } else if (particleName == "e-") {
    //electron
    pmanager->AddProcess(new G4MultipleScattering, -1,  1, 1);
    //standard geant4:
    //pmanager->AddProcess(new G4eIonisation, -1,  2, 2);
    //pmanager->AddProcess(new G4eBremsstrahlung, -1, -1, 3);
    
    // Low energy:
    G4LowEnergyIonisation* LeIoprocess = new G4LowEnergyIonisation("IONI");
    LeIoprocess->ActivateAuger(true);
    LeIoprocess->SetCutForLowEnSecPhotons(0.1*keV);
    LeIoprocess->SetCutForLowEnSecElectrons(0.1*keV);
    pmanager->AddProcess(LeIoprocess,-1,2,2);
    //pmanager->AddProcess(new G4LowEnergyIonisation, -1,  2, 2);
    
    G4LowEnergyBremsstrahlung* LeBrprocess = new G4LowEnergyBremsstrahlung();
    pmanager->AddProcess(LeBrprocess, -1, -1, 3);
    pmanager->AddProcess(new G4StepLimiter,-1,-1,3);
    //pmanager->AddProcess(new G4LowEnergyBremsstrahlung, -1, -1, 3);
    
    // Penelope:
    // pmanager->AddProcess(new G4PenelopeIonisation, -1, 2, 2);
    // pmanager->AddProcess(new G4PenelopeBremsstrahlung, -1, -1, 3);
    
  } else if (particleName == "e+") {
    //positron
    pmanager->AddProcess(new G4MultipleScattering, -1,  1, 1 );
    // standard Geant4 and Low energy
    pmanager->AddProcess(new G4eIonisation, -1,  2, 2 );
    pmanager->AddProcess(new G4eBremsstrahlung, -1, -1, 3 );
    pmanager->AddProcess(new G4eplusAnnihilation,  0, -1, 4 );
    pmanager->AddProcess(new G4StepLimiter(),  -1, -1, 3 );
    //Penelope:
    //pmanager->AddProcess(new G4PenelopeIonisation , -1,  2, 2 );
    //pmanager->AddProcess(new G4PenelopeBremsstrahlung, -1, -1, 3 );
    //pmanager->AddProcess(new G4PenelopeAnnihilation,  0, -1, 4 );
\end{verbatim}  
\normalsize

\subsection{Simulation output}
Again, as in other NPTool simulations, the results of the simulations are in the ROOT format 
and the output file is stored in the \$NPTool/Output/Simulation directory. If the PARIS 
geometry input file includes the NaI Shield, the output ROOT file contains four classes:

\begin{itemize}
   \item {TInitialConditions:}
      This class records all the information concerning the event generator
      such as the vertex of interaction, the angles of emitted particles in 
      the center of mass and laboratory frames...

   \item {TInteractionCoordinates:}
      Although this class appears in the tree, it is not in use with the PARIS array alone. 
      Please, see GASPARD documentation for more details.

   \item {TParisData:}
      This class stores the results of the simulation for the cluster and single phoswich 
      detector. Independently of the number and shape of the detectors involved in the 
      geometry, only {\it one} class is created for the whole PARIS detector. For each event,
      the energy for each layer of scintillators is recorded. The time, detector number and 
      crystal number information will be added in future release. At the moment, this is 
      equivalent of having a perfect add-back algorithm between crystals within each layer.
      
   \item {TShieldData:}
      This class stores the results of the simulation for Shield detector. 
      Independently of the number and shape of the 
      detectors involved in the geometry, only {\it one} class is created for 
      the whole shield detector. For each event, the energy is recorded.
      The time, detector number and crystal number information will be added in future 
      release. At the moment, this is equivalent of having a perfect add-back algorithm 
      between the shield crystals.
\end{itemize}


\section{NPAnalysis}

\subsection{General}

Only a very preliminary analysis code is provided with this release. 

\subsection{Paris}
The analysis code for the Paris array is in the \$NPTool/NPAnalysis/Paris
directory. For the moment the main feature is the photopeak efficiency as function of the 
$\gamma$-ray incoming energy. The photopeak efficiency in each layer of the array is 
calculated as well as for the NaI sheild. The full photopeak efficiencies, with and without 
add-back correction between the different layers, are also determined. 

\subsubsection{Running the analysis}
To run NPAnalysis the following command line should be executed in the 
\$NPTool/NPAnalysis/Paris directory:

\begin{verbatim}
   ./Analysis -D yyy.detector -E xxx.reaction -R RunToTreat.txt
\end{verbatim}

where xxx.reaction is the input file describing the event generator used in 
NPSimulation and yyy.detector is the input file describing the detector geometry
used in NPSimulation. All these input files are based on keywords and can be found 
in the \$NPTool/Inputs subdirectories. The RunToTreat.txt file contains the
name of the files (either from NPSimulation or from real experiment) which should
be analysed. The name of the tree should also be specified. An example 
of such a file is given here:

\begin{verbatim}
   TTreeName
           SimulatedTree
   RootFileName
          ../../Outputs/Simulation/myResult.root
   %       ../../Outputs/Simulation/mySimul.root
\end{verbatim}


\subsubsection{Results of the analysis}
The results of the analysis are stored in a ROOT file in the \$NPTool/Output/Analysis
directory.


\subsubsection{Structure of the analysis}
******* to be documented *********


\end{document}

