\documentclass[a4paper,12pt]{article}
\usepackage[margin=2cm]{geometry}

\usepackage[T1]{fontenc}
\usepackage [isolatin]{inputenc}
\usepackage{graphicx}
\usepackage{listings}

\begin{document}

\title{\emph{Gaspard Documentation}}
\author{Nicolas de S\'er\'eville}

\maketitle 
\pagebreak
\tableofcontents
\pagebreak


\section{Introduction}
The Gaspard project is developed within the NPTool framework which is a 
modular package allowing to perform Geant4 simulations and to analyse the 
results of the simulations. It is strongly encouraged to read the general 
NPTool documentation that you can find in this directory.

Concerning the status for Gaspard simulations, the charged particles tracker
is now finished and the coupling with the PARIS gamma-ray calorimeter is
done. Both the tracker and the calorimeter have different available geomtries.


\section{NPSimulation}
\subsection{Specificity of Gaspard}
The Gaspard tracker detector, even if it is made of several detectors of
different shapes (square, trapezoid, annular, ...), is considered as {\it one} 
detector from the NPSimulation point of view. The Gaspard tracker detector 
is described in the GaspardTracker class defined in the GaspardTracker.\{hh,cc\} 
files. Since the Gaspard tracker detector is registered in the 
DetectorConstructor.cc file it is available for NPSimulation.

In order to manage the different detector shapes (square, trapezoid, annular, 
...) of the Gaspard tracker, the GaspardTracker class holds a vector of 
GaspardTrackerModule objects from which are, and should be, deriving all 
the different shapes (GaspardTrackerSquare, GaspardTrackerAnnular,
GaspardTrackerTrapezoid and GaspardTrackerDummyShape classes).


\subsection{Running the simulation}
To run NPSimulation the following command line should be executed: 

\begin{verbatim}
   Simulation -E xxx.reaction -D yyy.detector
\end{verbatim}

where xxx.reaction is an input file describing the event generator and
yyy.detector is an input file describing the detector geometry. All these
input files are based on keywords and can be found in the \$NPTool/Inputs 
subdirectories.

\subsubsection{Event Generators}
All the different kind of event generator files as well as their 
respective keywords are described in the general NPTool documentation. 

\subsubsection{Detector Configurations}
The keywords associated to the detector geometry file are different for 
each detector. In case of the Gaspard tracker detector an example with 
all the detector shapes available at the moment is given in the following:

\begin{verbatim}
   %%%%%%%%%%%%%%%%%%%%%%%%%%%%%%%%%%%%
   GaspardTracker
   %%%%%%%%%%%%%%%%%%%%%%%%%%%%%%%%%%%%
   GPDAnnular
           Z=      -156.5
           RMIN=   16
           RMAX=   52
           FIRSTSTAGE= 1
           SECONDSTAGE= 0
           THIRDSTAGE= 1
           VIS= all
   %%%%%%%%%%%%%%%%%%%%%%%%%%%%%%%%%%%%
   GPDTrapezoid
           X1_Y1=           45.64   34.43  -146.50
           X128_Y1=         91.09   79.82   -91.36
           X1_Y128=        120.84    8.00   -91.36
           X128_Y128=       56.59    8.00  -146.50
           FIRSTSTAGE= 1
           SECONDSTAGE= 0
           THIRDSTAGE= 1
           VIS= all
   %%%%%%%%%%%%%%%%%%%%%%%%%%%%%%%%%%%%
   GPDSquare
           X1_Y1=          49.1    66.08   -135.41
           X128_Y1=        -48.9   66.22   -135.41
           X1_Y128=        -48.8   135.51  -66.1
           X128_Y128=      49.2    135.36  -66.1
           FIRSTSTAGE= 1
           SECONDSTAGE= 0
           THIRDSTAGE= 1
           VIS= all
   %%%%%%%%%%%%%%%%%%%%%%%%%%%%%%%%%%%%
   GPDDummyShape
           THETA= 90
           PHI= 90
           R= 100
           BETA= 0 0 0
           FIRSTSTAGE= 1
           SECONDSTAGE= 1
           THIRDSTAGE= 1
           VIS= all
\end{verbatim}

In order to declare a Gaspard tracker detector in NPSimulation, the key
word {\it GaspardTracker} should be specified in the geometry file. It
should then be followed by other keywords concerning the different
detectors present in the tracker. Such keywords available at the moment
are:

\begin{itemize}
   \item {GPDAnnular}
   \item {GPDTrapezoid}
   \item {GPDSquare}
   \item {GPDDummyShape}
\end{itemize}

Each keyword corresponds to a detector shape which have its own set of
keywords which is used to position the detector in the world volume
and to define the structure of the detector (basically the number of 
layers of the detector).

To position the detectors two possibilities exist. Either the cartesian 
coordinates (x,y,z) of each detector's corner are specified with the 
keywords {\it X1\_Y1}, {\it X1\_Y128}, {\it X128\_Y1} and {\it X128\_Y128}
(case of GPDTrapezoid), either the spherical coordinates of the detector's 
center are specified with the keywords {\it R}, {\it THETA} and {\it PHI} 
(case of GPDDummyShape). While the first solution is very helpful when 
working with the mechanical engineers, the second solution is useful when 
investigating new geometries.

In case of GPDAnnular only the position on the z-axis is given through
the keyword {\it Z}. Other keywords such as {\it RMIN} and {\it RMAX} 
are defined but have no effect. This should be changed in the future.

Concerning the structure of the detector, all the detectors have the 
possibility to have up to three layers of silicon. Each layer can be 
activated independently using the keywords {\it FIRSTSTAGE}, 
{\it SECONDSTAGE} and {\it THIRDSTAGE}. 


\subsection{Results of the simulation}
The results of the simulation are in the ROOT format and the output file 
is stored in the \$NPTool/Output/Simulation directory. The output ROOT file 
contains three classes:
\begin{itemize}
   \item {TInitialConditions:}
      This class records all the information concerning the event generator
      such as the vertex of interaction, the angles of emitted particles in 
      the center of mass and laboratory frames...
   \item {TInteractionCoordinates:}
      This class mainly records the cartesian and spherical coordinates of 
      interaction between a particle and a detector.
   \item {TGaspardTrackerData:}
      This class stores the results of the simulation for the tracker part 
      of the Gaspard detector. Independently of the number and shape of the 
      detectors involved in the geometry, only {\it one} class is created for 
      the whole Gaspard tracker detector. For each event the strips number 
      are recorded as well as the energy and time for the layers which are 
      involved in the telescope.
\end{itemize}


\subsection{Adding a new detector shape to Gaspard}
A special class (GaspardTrackerDummyShape) has been created to show how
to define a new module for the Gaspard Tracker. This class describes a
simple 5x5 cm$^2$ square telescope made of three layers of silicon which  
has been used for some preliminary studies of the tracker. So, when 
considering adding a new module to the Gaspard Tracker, please do not use
this class but create your own instead. However, for the explanations the 
GaspardTrackerDummyShape case will be considered.

When adding a new detector you need to follow several steps:

\subsubsection{Definition of an index for the detector}
Since the results of the simulation are stored in an unique data class 
(GaspardTrackerData) dealing with different module types, it is 
necessary to give an absolute number to each module. This is managed in
the GaspardTrackerModule class where there is a map (m\_index) which
associates a name (the module type identifier) with an integer (the value 
of the index).

To add a new detector it is just needed to add in the {\it InitializeIndex()}
method a line such as:

\begin{verbatim}
   m_index["DummyShape"] = 1000;
\end{verbatim}

\subsubsection{Definition of the geometry and its readout}
This is done in the GaspardTrackerDummyShape.\{h,cxx\} files. Concerning
the geometry it is defined in the {\it VolumeMaker()} method and the 
the positioning of the module is done in the {\it ConstructDetector()} 
method. 

Concerning the readout of the geometry it is done in the {\it ReadSensitive()} 
method and it is based on the method of the scorers available in G4. The 
scorers associated to the Gaspard tracker are declared in the {\it 
InitializeScorers()} method.

\subsubsection{Definition of the scorers}
If the scorers are declared in the {\it InitializeScorers()} method they should
be defined in the GaspardScorers.\{hh,cc\} files. All basic scorers to record
energy, time of flight and detector number are already implemented so when a new
detector is added to Gaspard tracker there is nothing to add from this point
of vue. However scorers determining the strip number for the front and back
side of the silicon detector's first stage (if double-sided) should be implemented.
In case of the GaspardTrackerDummyShape class this corresponds to the two classes
GPDScorerFirstStageFrontStripDummyShape and GPDScorerFirstStageBackStripDummyShape.

\subsubsection{Integration in GaspardTracker}
In order to make the GaspardTracker detector aware of the GaspardTrackerDummyShape
module it has to be registered in the {\it ReadConfiguration()} method of the 
GaspardTracker class. Don't forget to include the GaspardTrackerDummyShape.hh 
header in the GaspardTracker.cc file. Then, in the GaspardTrackerDummyShape class 
the keywords used when the geometry file is read should be defined in the {\it 
ReadConfiguration()} method.


\section{NPAnalysis}
\subsection{General}
A set of general ROOT macros are available in the \$NPTool/NPAnalysis/macros 
directory. You can for example obtain some control plots about the shooting
conditions of the random variables. You can also calculate the geometrical 
efficiency of your setup.

The macros in this directory should be independant of the setup which is simulated.
Specific macros to Gaspard tracker should be placed in the 
\$NPTool/NPAnalysis/Gaspard/macros directory.


\subsection{Gaspard}
The main analysis tool for the Gaspard tracker is in the \$NPTool/NPAnalysis/Gaspard
directory. For the moment the main feature is the reconstruction of the excitation 
energy.

\subsubsection{Running the analysis}
To run NPAnalysis the following command line should be executed:

\begin{verbatim}
   ./Analysis -E xxx.reaction -D yyy.detector -R RunToTreat.txt
\end{verbatim}

where xxx.reaction is the input file describing the event generator used in 
NPSimulation and yyy.detector is the input file describing the detector geometry
used in NPSimulation. All these input files are based on keywords and can be found 
in the \$NPTool/Inputs subdirectories. The RunToTreat.txt file contains the
name of the files (either from NPSimulation or from real experiment) which should
be analysed. The name of the tree should also be specified. An example 
of such a file is given here:

\begin{verbatim}
   TTreeName
           SimulatedTree
   RootFileName
   %       ../../Outputs/Simulation/mySimul.root
           ../../Outputs/Simulation/132Sndp_3p3_10MeVA_T0_B1_E0_S05mm.root
   %       ../../Outputs/Simulation/134Snpt_1h9_10MeVA_T1_B1_E0_S05mm.root
\end{verbatim}


\subsubsection{Results of the analysis}
The results of the anaysis are stored in a ROOT file in the \$NPTool/Output/Analysis
directory. For the moment the main feature available is the reconstruction of the 
excitation energy.


\subsubsection{Structure of the analysis}
The analysis package now deals with Gaspard trackers including different shapes.
The main analysis program is the \$NPTool/NPAnalysis/Gaspard/src/Analysis.cc
file where the user is in charge to code the specific functionalities he is
interested in. This includes to treat cases with multiplicity greater than one,
calculate the excitation energy, check the effect of beam tracker detector
positioning resolution, ...

However, all the basic treatments of the analysis are done in \$NPTool/NPLib/GASPARD.
These basic treatments calculate the total energy deposited in the telescope and
associate for each pixel number the cartesian coordinates of the pixel middle.

The structure in \$NPLib/GASPARD is very similar to the NPSimulation structure
for Gaspard. The GaspardTracker.\{h,cxx\} class derives from the VDetector class
and holds ($i$) the GaspardTrackerData object to be analysed, ($ii$) the
GaspardTrackerPhysics object with the results of the basic treatement and ($iii$)
a map which associates the detector number read in the geometry file with a pointer
to the GaspardTrackerModule class. In this way a Gaspard tracker geometry with any 
kind of shapes is analyzed correctly. Each available shape (square, trapezoid, annular,
dummyshape, ...) has its own class GaspardTracker*Shape*.\{h,cxx\} deriving from 
the GaspardTrackerModule class. Each class basically applies the same stripping scheme 
as defined in the simulation in order to obtain the cartesian coordinates of the center 
of each pixel. Each class have its own treatement method to calculate the total energy
deposited in each module.


IMPORTANT NOTE: for the moment, only multiplicity one are considered.



\end{document}

